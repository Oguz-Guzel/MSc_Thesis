2012 yılında, İsviçre’nin  Cenevre  kenti  ile  Fransa  sınırında  konuşlanmış olan Avrupa Nükleer Araştırma Merkezi’nin (CERN) Büyük Hadron Çarpıştırıcısı (LHC) etrafındaki CMS  ve  ATLAS  deneyleri, 125 GeV kütlesinde bir parçacığın gözlendiğini açıkladı. İlk defa varlığı deneyler tarafından kanıtlanan bu parçacık, 1964 yılında Peter Higgs ve 5 diğer bilim adamı tarafından ortaya atılan Higgs bozonuydu. Bir sene sonra Peter Higgs ve François Englert Nobel Fizik Ödülü'ne layık görüldüler. 2012'den itibaren, elektro-zayıf simetri kırınımı mekanizmasını açıklayan ve Standart Model ötesi kuramlar için önemli bir parçacık olan Higgs bozonunun tüm özellikleri detaylı bir şekilde çalışılmaya başlandı.

27 km uzunluğunda çevreye sahip çember şeklinde bir yeraltı tünelini kullanan LHC deneyi, 4 adet büyük detektör barındırmaktadır. Bu detektörler; ATLAS, ALICE, CMS ve LHCb detektörleridir. Bunlardan ikisi, ATLAS ve CMS, 13 TeV kadar yüksek bir kütle merkezi enerjisinde çarpışan protonlardan saçılan parçacıkları gözlemleyerek, Higgs bozonu dahil olmak üzere diğer tüm temel parçacıkları araştırıyor. LHC deneyi, 2027 yılı sonrası için 14 TeV'lik bir çarpışma enerjisine çıkmayı ve toplam lüminositeyi 3000 $fb^{-1}$ değerine kadar yükseltmeyi planlarken CMS deneyi de, çarpışma noktasının etrafındaki kapsama alanını artırmak gibi çeşitli güncellemelerle geliştiriliyor.

Higgs bozonu çifti üretimi LHC'deki proton-proton çarpışmalarında görece yüksek bir kesit alanına sahiptir. 13 TeV enerjide gluon-gluon füzyonu üretim yöntemi yaklaşık 30 $fb$'dir. Yüksek enerjideki proton-proton çarpışmalarında Higgs bozonu çifti üretiminin önemi ise, Higgs potansiyelinin şeklini belirleyen Higgs tri-lineer self-etkileşme terimine, vektör bozon üretim yöntemi ile doğrudan erişiminin olmasıdır. LHC deneyinde üretilen Higgs bozonu çiftleri ATLAS ve CMS deneyleri tarafından bir çok bozunma kanalında çalışılmıştır. Fakat Yüksek Lüminosite LHC programı (HL-LHC), yani LHC deneyinin 2027 ve sonrası için planlanan güncellenmiş halindeki Higgs bozonu çifti üretimi henüz tüm bozunma kanallarında çalışılmamıştır.

Bu tezde, HL-LHC programı kapsamında yapılması planlanan güncelleme çalışmaları sonrasındaki lüminositede Higgs bozonu çifti üretimi, gluon-gluon füzyonu üretim yöntemi ile CMS detektörünün Faz-2 için güncellenmiş halinde, proton-proton çarpışması simülasyonları ile çalışılmıştır. 1. bölümde, parçacık fiziğinin başarılı teorisi Standart Model'e giriş niteliğinde açıklamalar verilmiştir. Standart Model Lagrange denklemleri incelenmiş, temel kuvvetler, parçacıklar ve özellikle Higgs bozonundan ve Higgs bozonu çift üretiminden bahsedilmiştir. Standart Model ötesi araştırmalardan ve bu araştırmalarda Higgs bozonunun öneminden bahsedilmiştir.

2. bölümde, Büyük Hadron Çarpıştırıcısı deneyinin yapısı açıklanmış ve simülasyon programları ile davranışı taklit edilen Kompakt Müon Solenoidi (CMS) detektörünün özellikleri ve alt detektörleri incelenmiştir. CMS detektörünün Faz-2 Güncelleme Çalışmaları ve HL-LHC programı açıklanmıştır.

Higgs bozonu çiftinin \wwgg bozunma kanalındaki analizi  3. bölümde detaylıca incelenmiştir. Öncelikle analiz stratejisinden bahsedilmiş ve simülasyon ile üretilen proton-proton çarpışma verileri; bu verilerin üretiminde kullanılan programlar ve bu programların arka planında çalışan metotlar, sinyal ve artalan çarpışma proseslerinin detayları ile CMS detektörünün simülasyonu için kullanılan bilgisayar programının (\textsc{Delphes}) özellikleri açıklanmıştır. Analizde kullanılan fizik objeleri; fotonlar, leptonlar, jetler ve kayıp enerji, CMS detektörünün kapsama alanı içerisinde ve yeterli enerjiye sahip olacak şekilde tanımlanmıştır. Higgs bozonu çifti üretiminin istenilen kanallarda (\wwgg, \ttgg) incelenmesini sağlamak amacıyla çarpışma olaylarına belirli kriterler uygulanmıştır. Olay seçimi adı verilen bu işlem ve fizik objesi tanımlamaları, \textsc{Bamboo} adı verilen Python temelli bir analiz kütüphanesi yardımıyla yapılmıştır.

\textsc{TensorFlow} makine öğrenmesi kütüphanesi ile \textsc{Keras} arayüzü kullanılarak, analiz için 2 adet yapay sinir ağı geliştirilmiştir. Bu çalışmadaki modelin, analizi öğrenip tahminlerde bulunması için fizik objelerinin; olay başına miktarları, dik-eksen momentumları (\emph{transverse momentum}), dik eksendeki ($\eta$) ve azimut eksenindeki ($\phi$) açı değerleri, fotonlar ve tau leptonları için; sabit kütlelerinin (\emph{invariant mass}) enerji ve dik eksen momentumlarına oranları gibi değerler girdi olarak verilmiştir. Bu girdileri kullanarak 0 ile 1 arasında yapay zeka ağı puanı dağılımı çıkarılmıştır. Bu dağılımda sinyal ile artalan proseslerinin birbirinden ayrılması ROC eğrileri ve istatistiksel önem değerleri olay bazında hesaplanarak sağlanmış ve yapay zeka ağı puanının farklı aralıklardaki değerleri için kategoriler oluşturulmuştur. Belirlenen bu kategoriler analiz kodlamasında yeni birer seçim olarak uygulanmış ve bu seçim sonrasında elde edilen foton çifti sabit kütlesi dağılımları çıkarılmıştır.

Sistematik belirsizlikler 3. bölümün son konusu olarak açıklanmıştır. Bu kısımda CMS ve ATLAS deneyleri tarafından önerilen deneysel ve teorik sistematik belirsizlikler uygulanmıştır. Deneysel belirsizlikler; lüminosite, foton çifti tetiklemesi, foton çifti sabit kütlesi, foton ID, elektron ID, muon ID, tau ID ve jet enerji ölçeği belirsizlikleridir. Teorik belirsizlikler ise ,kesit alanı belirsizlikleri, QCD skalası ve parton dağılım fonksiyonları (PDF) olarak her bir proses veri setine uygulanmıştır. Farklı olarak sinyal proseslerine top kuark kütlesinin belirsizliği de eklenmiştir. Daha sonra \textsc{Higgs Combine Tool} adı verilen, Higgs analizleri için özelleştirilmiş bir analiz kütüphanesi kullanılarak istatistiksel önem düzeyleri (\emph{statistical significance}) çıkarılmıştır. \textsc{Higgs Combine Tool} kütüphanesine, foton çifti sabit kütlesi dağılımı girdi olarak verilmiş ve sistematik belirsizlikler bu aşamada uygulanmıştır.

Sonuçların sunulduğu 4. bölümde ise, anlamlılık düzeyleri ve yakınlık taramaları verilmiş ve analiz açısından önemleri tartışılmıştır. Bazı son bozunma durumlarındaki foton çifti kütlesi dağılımları artalan ve sinyal proses sonuçlarına uyarlanıp grafik olarak sunulmuştur. Ekler bölümünde, analizdeki kullanılan tüm veri setleri listelenmiş, tüm bozunma durumlarında seçilim yapılmadan önceki ve sonraki olay sayıları ve seçilim verimleri yüzdelik olarak tablo halinde verilmiştir. Ayrıca yapay zeka sinir ağlarına girdi olarak kullanılan tüm dağılımlar verilmiş, ve performans grafikleri eklenmiştir.
