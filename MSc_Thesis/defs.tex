%============================================== NEW INPUTS ==============================================
\usepackage[T1]{fontenc} 				% For Turkish characters
\usepackage[utf8]{inputenc}  			% For Turkish characters
\usepackage[open,openlevel=1]{bookmark} % It is used to have bookmarks in the PDF file created
% Abbreviations & Nomenclature
%\usepackage[refpage]{nomencl}
%\makenomenclature
%\usepackage{textcomp}
%\usepackage{array}
%\usepackage{lscape}
%\usepackage{listings} % Required for insertion of code
%\usepackage{xcolor}
%\usepackage[colorlinks=true,linkcolor=blue,urlcolor=black,bookmarksopen=true]{hyperref}
%\usepackage{biblatex}

%====================================== ITU NEW PACKAGES INCLUDED ======================================
\usepackage{color}
\usepackage{times}
\usepackage{amssymb,amsmath,mathptmx,amsbsy,bm}
\usepackage{caption}            
%\usepackage{floatflt}
%\usepackage[dvipdfm]{graphicx}
\usepackage{graphics}
\usepackage{wrapfig}
\usepackage{epsfig}
\usepackage{enumerate}
\usepackage{rotating}
\usepackage{multirow}					% Multirow in tables
%\usepackage{subfigure} 				% This is obsolete, therefore use subcaption package instead - SBÖ
\usepackage{colortbl}
\usepackage{pstricks}
\usepackage{pst-plot}
\usepackage{cite}
\usepackage{latexsym}
\usepackage{xcolor}
%\usepackage{subeqn}
%\usepackage{hyperref}
\usepackage{hyperref}\hypersetup{hidelinks} % The line is added for hiding the links in document.
%\usepackage{url}
%\usepackage{fixltx2e} % Bu paketi sembollerde text ler için subscript yazmakta yardımcı olması için ekliyoruz.
%\usepackage{ulem} 			% That does destroy Dedication and Bibliography (Use the below version xpatch) - SBÖ 
\usepackage{xpatch} 		% Added for TOC dot flush to the page number - SBÖ
\usepackage[normalem]{ulem} % For special underline tricks at the top of kapak pages - SBÖ 
\usepackage[bottom,multiple]{footmisc} 	% Add hang to align to the left - SBÖ (bottom,flushmargin)
%\usepackage{fnpos} 					% Another footnote position package - SBÖ
\setlength{\skip\footins}{1cm} 			% Placement of the last text from the top of the Footnote - SBÖ
%\setlength{\skip\footins}{1\baselineskip}
\setlength\footnotemargin{.35em} 		% Footnote indentation the first line from left margin - SBÖ
\addtolength{\footnotesep}{1mm} 		% Distance change to 1 mm line with footnote text at the bottom - SBÖ
%\setlength{\footnotesep}{.5\baselineskip}
\usepackage{enumitem} 					% Used for bullets in the resume to flush left as in the word template - SBÖ
\renewcommand\labelitemi{\normalsize$\bullet$} % Set the bullet size similar to the word template - SBÖ
%\makeFNbottom 							% Used with fnpos package - SBÖ
\usepackage{pdfpages} 					% http://ctan.org/pkg/pdfpages - SBÖ
%\usepackage{fancyhdr} 					% http://ctan.org/pkg/fancyhdr - SBÖ
%\usepackage{geometry}
\usepackage{tikzpagenodes} 				% For landscape page numbering - SBÖ
%\usepackage{setspace} 					% Provides support for setting the spacing between lines - SBÖ
%\usepackage{showframe} 				% http://ctan.org/pkg/showframe - To show the margins in a frame on pages - SBÖ
\usepackage{etoolbox}					% http://ctan.org/pkg/etoolbox - For removing default 50pt TOx stuffs from top - SBÖ
\makeatletter 															% Used with etoolbox - SBÖ
\patchcmd{\@makechapterhead}{\vspace*{50\p@}}{}{}{}						% Removes space above \chapter head
\patchcmd{\@makeschapterhead}{\vspace*{50\p@}}{\vspace*{21.5mm}}{}{}	% Removes space above \chapter* head and add 21.5mm
\makeatother
\usepackage{longtable} 	% Include long tables in the text spreading more than one page - SBÖ
\usepackage{hhline} 	% If desired to eliminate hline in the tables - SBÖ
\usepackage{siunitx}
%\usepackage{subcaption} % Make subfigure as Figure 1.1a style - SBÖ
\usepackage[list=true,listformat=simple,position=below]{subcaption} % Make subfigure as Figure 1.1a style - SBÖ
% Subfigure caption settings - SBÖ
\DeclareCaptionLabelFormat{subfig}{\normalsize\figurename #1~\arabic{chapter}.\arabic{chapter}\alph{subfigure} :}
%\DeclareCaptionListFormat{subfigure}{\arabic{chapter}.\arabic{chapter}\alph{subfigure}}
\captionsetup[subfigure]{labelformat=subfig, size=normalsize}

% Bold Equation Number, Unbold Reference
\makeatletter
\def\tagform@#1{\maketag@@@{\bfseries(\ignorespaces#1\unskip\@@italiccorr)}} % All bold in eq. number incl. parentheses 
%\renewcommand{\eqref}[1]{\textup{{\normalfont(\ref{#1}}\normalfont)}}
\renewcommand{\eqref}[1]{\textup{\bf(\ref{#1})}} % All bold in eq. referencing with parentheses - SBÖ
\makeatother

%%%%%%% OGUZ %%%%%%%%%
\usepackage{listings}
\usepackage{xcolor}

%New colors defined below
\definecolor{codegreen}{rgb}{0,0.6,0}
\definecolor{codegray}{rgb}{0.5,0.5,0.5}
\definecolor{codeblue}{RGB}{51,51,255}
\definecolor{backcolour}{RGB}{255,255,255}

%Code listing style named "mystyle"
\lstdefinestyle{mystyle}{
  backgroundcolor=\color{backcolour}, commentstyle=\color{codegreen},
  keywordstyle=\color{blue},
  numberstyle=\tiny\color{codegray},
  stringstyle=\color{codeblue},
  basicstyle=\ttfamily\footnotesize,
  breaklines=true,
  breakatwhitespace=true,
  numbers=left,
  numbersep=4pt,    
}
\lstset{style=mystyle}

\usepackage{xspace}
\newcommand{\ttgg}{\ensuremath{\tau\tau\gamma\gamma}\xspace}
\newcommand{\wwgg}{\ensuremath{WW\gamma\gamma}\xspace}
\newcommand{\zzgg}{\ensuremath{ZZ\gamma\gamma}\xspace}
\newcommand{\mgg}{\ensuremath{m_{\gamma\gamma}}\xspace}

\def\pt{$p_{T}$\xspace} % pT
\def\kl{$\kappa_\lambda$\xspace} % kappa lambda
\newcommand{\Lag}{\mathcal{L}}

%%%%%%% OGUZ %%%%%%%%%

\def\be{\begin{equation}} %
\def\ee{\end{equation}}%
\def\bse{\begin{subequations}}%
\def\ese{\end{subequations}}%

