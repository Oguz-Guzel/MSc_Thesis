\chapter{INTRODUCTION TO THE STANDARD MODEL}\label{Ch1}

The Particle Physics is probing the smallest objects that are known as elementary particles and tries to extend our knowledge of the subatomic world. These elementary particles are accelerated, collided and detected at very high energies in the experiments around the globe - one of them being the largest experimental setup ever built for science - and studied after being detected. The Standard Model of particle physics (SM) is the theory of the fundamental interactions in this pursuit. It is a quantum field theory developed with the contribution of many scientists around the globe mainly in the second half of the 20$^{th}$ century and, over the last few decades, it has been shown to be an accurate description of the picture. It describes all known particles but is a mathematical description of three of the four known fundamental forces of the nature. These are the electromagnetic interaction, the weak and the strong nuclear interactions. The gravitational force, due to difficulties in combining general relativity with quantum mechanics, does not take place in the Standard Model. The gravity is known to be 10$^{40}$ times weaker than the electromagnetic force, thus its effects are expected to be negligible in the theory.

The Standard Model, being a quantum field theory, was formulated in the 1960's. It is grounded on the mathematical concept of the local gauge invariance under the symmetry groups of its Lagrangian equations. It has predicted successfully a wide range of phenomena which have been observed in the experiments around the globe. The discovery of the $W^\pm$ and $Z^0$ bosons \cite{SPS-1, SPS-2, SPS-3, SPS-4, SPS-5} at the CERN's Super Proton Synchrotron (SPS) by the UA1 and UA2 Collaborations in 1983, the observation of the top quark at the FNAL's Tevatron by D$\Phi$ and CDF Collaborations\cite{fnal-1, fnal-2} in 1995 and the observation of the Higgs boson at CERN's LHC experiment\cite{HiggsCMS,HiggsATLAS} by CMS and ATLAS Collaborations and many other discoveries along with verifications can be shown as the biggest successful predictions of the SM.

However, this successful theory had a flaw; it could not explain the reason for the masses of the $W^\pm$ and $Z^0$ bosons. The solution came from R. Brout, F. Englert and P. Higgs in 1964 called the Brout-Englert-Higgs (BEH) Mechanism\cite{Higgs1964, BroutEnglert, Guralnik1964}. They introduced a complex scalar field into the SM and assumed a spontaneous symmetry breaking, hence giving mass to the vector bosons, also to the leptons while preserving the unitarity of the SM theory. This mechanism foresaw the existence of a scalar boson field. Later on, the BEH mechanism was incorporated in the electroweak model by Salam and Weingberg\cite{salam} in 1964, and proven renormalisable by 't Hooft and Veltman in 1972\cite{thooft}. The predicted boson is called the Higgs boson and searched by high energy physicists in the experiments nearly for 50 years.

In 1995, the technical design report for the LHC was published\cite{lhc-tdr}. It involved constructing a new particle accelerator in the Franco-Swiss border in Geneva and was intended to explore the TeV energy scale upto centre-of-mass energies of 14 TeV. The details of the LHC is explained in \autoref{Ch2}. The LHC has lead the forefront of the experimental particle physics research for many years and resulted in the Higgs boson at a mass of 125 GeV which confirmed the BEH mechanism.

In the SM, the elementary particles of matter and the ones that carry forces between them is grouped into two, namely fermions and bosons. The distinction shows itself in their spin properties; fermions have a spin value of half an integer whereas the bosons have integer spin values. Besides, the fermions obey the Pauli exclusion principle meaning that they cannot be at the same quantum state, whereas the bosons do not obey the same rule. The first generation of fermions include up and down quarks, electron and its neutrino counterpart; and bosons include photons, W and Z bosons, gluons and the Higgs boson. When added the second and the third generation of particles - whose existence is one of the mysteries of nature - along with their anti-particles, they form the fundamental particles that are known today. A tabular form of these particles can be seen in~\autoref{SM-figure}.

\vspace{6pt}
\begin{figure}[ht]
	\centering
	\includegraphics[width=\textwidth]{SM.jpg}
	\vspace{6pt}
	\caption{The elementary particles of the Standard Model. The fermions and bosons are grouped in columns where the quarks, leptons, gauge bosons and scalar Higgs boson are shown in different colours. The three generations of matter are indicated with roman numerals. The mass, charge and spin values corresponding to each particle are indicated on the upper left of each box. The bosons are shown in faint yellow areas with which they interact with.}
	\label{SM-figure}
\end{figure}

Fermions make up the ordinary matter that we see around us everyday. A subgroup of fermions are the leptons and they include electron, muon, tau and their neutrino counterparts. Neutrinos have very small mass values compared to other leptons and quarks. They do not have electromagnetic charge which makes them obey only the weak force and they barely interact with matter. Flavours of leptons other than the neutrinos (electrons, muons and taus) have sizeable mass and charge and they are members of the three generations of matter.

The other type of leptons are the quarks. They have three generations as leptons do, and they include six different flavours, namely up, down, charm, strange, top and bottom quarks. They interact with electromagnetic, weak and strong nuclear forces since they have colour charges in addition to their hypercharge. 

Bosons are the mediator particles of the three forces described in the SM. The particles of this type are called the gauge bosons and they include the photon ($\gamma$), W$^{\pm}$ bosons, Z boson, gluons and the Higgs boson. The photons and gluons do not have mass where W bosons have about 80 GeV/c$^{2}$, Z boson has 91 GeV/c$^{2}$ and the Higgs boson has 125 GeV/c$^{2}$ of mass. Among these, only the W$^{\pm}$ bosons have electric charge, and only the Higgs boson has a spin value of zero making it a scalar boson where all other bosons are spin-1 gauge bosons, meaning that they are vector bosons.

The electromagnetic force, whose quantum field theory is established by the \emph{Quantum Electrodynamics (QED)}, is mediated by photons and acts only on the charged particles.  Almost everything we see in the daily life is thanks to the electromagnetic interactions. The carrier of this force, the photons, do not carry electric charge, nor self-interact. Since the photons have zero mass, their interaction range is unlimited.

\emph{The weak interaction}, which is responsible for the decay of the particles that is a flavour changing interaction, acts only on the fermions. It has a very short interaction range. The neutral and the charged current interactions are mediated by the vector bosons of the this force, the Z boson and the W$^{\pm}$ bosons, respectively. The Higgs boson in this picture, plays the role of generating the masses of the W$^{\pm}$, and Z bosons through the Higgs mechanism~\cite{Higgs1964, BroutEnglert, Guralnik1964}, and of the quarks and leptons (except neutrinos) through Yukawa interaction\cite{Weinberg1967}.

\emph{Quantum Chromodynamics (QCD)} is the theory of the strong interaction and describes the interaction between quarks mediated by the massless gluons. It has an interaction range of about $10^{-15}$ m. Unlike the electromagnetic interaction, the gauge bosons of the strong force possess the charge of the interaction, namely colours, hence can couple to itself. There are 3 types of colour; red, green and blue. Each quark in the universe carry one of these 3 colours - they carry anti-colour if they are anti-quarks - and gluons can be thought of colour carrier particles.

A phenomenon called \emph{the colour confinement} states that colour-charged particles cannot be isolated, meaning that only colour-neutral particles can be observed in nature\footnotemark. This results in that gluons carry a pair of charge consisting a colour and an anti-colour. Also, the hadrons are colour-neutral in two ways; i) with a pair of colour and anti-colour or ii) with 3 different types of colour or anti-colour. The first combination makes up mesons, consisting of a quark pair (a quark and an anti-quark) and the second combination forms baryons.

\footnotetext{~Below the Hagedorn temperature ($T_H$) of about 0.15 GeV. At the energies higher than $1.7x10^{12}$ K, hadrons become unstable and it can be thought of the boiling point of the hadronic matter~\cite{Hagedorn2016}.}

Some of the predictions of the SM are observed in the experiments, and there is no other particle or force is found beyond the scope of the SM. However, the SM does not provide answers to the unsolved problems in the fundamental physics such as the non-zero masses of the neutrinos\cite{neutrino-mass}, or the dark matter and the dark energy\cite{PlanckCol} which are the dominated energy content of the universe. Therefore the SM is seen as an effective field theory.

\section{The Standard Model Lagrangian and the Higgs Mechanism}
\label{theSMandHiggs}

The Standard Model describes the mentioned fundamental interactions and elementary particles in a single Lagrangian. It is often considered in two parts: the strong sector offers a description for the particles with colour charges, while the electroweak part consists of the electromagnetism and the weak force. The SM is a renormalised gauge field theory with the $ SU(3)_C \otimes SU(2)_L \otimes U(1)_Y$ gauge form and the charges are colour, weak isospin and hypercharge, respectively. The SU(2)$_L$ and U(1)$_Y$ groups mix and the W$^{\pm}$, Z and $\gamma$ bosons are created where $SU(3)_C$ gauge group describes the strong force. The Lagrangian here, does not involve the particle masses but they are introduced to the theory via \emph{the spontaneous symmetry breaking} of the $SU(2)_L \otimes U(1)_Y$ group, that is the electroweak gauge group, which we will address after studying the SM Lagrangian.

Mathematical interpretation of the SM is provided by \emph{the Quantum Field Theory (QFT)}, where each particle is represented by a quantum field that is pervaded across the space-time. The behaviour of the the fundamental fields and the quantum states are determined by the Lagrangian density (usually called \emph{the Lagrangian}). Most of the field theories normally starts with defining a set of symmetries of the system and continues with writing down the renormalisable Lagrangian of the particles or fields that obey these symmetries. The QFT also follows this path; it consists of translational symmetry, rotational symmetry and a boost symmetry that is the invariance of an inertial reference frame. The internal symmetry that defines the SM is a local gauge symmetry of the $ SU(3) \otimes SU(2) \otimes U(1)$ group, where U(1) acts on B and $\phi$, SU(2) acts on W and $\phi$ and SU(3) acts on G fields. The fermion fields also, depending on their charge, transform under these symmetries. The Lagrangian of the SM can be interpreted in three parts; Quantum Chromodynamics, Electroweak theory and the Higgs Mechanism.

\subsection{Quantum Chromodynamics}

Starting from the free-field Dirac Lagrangian density for a massless spin-1/2 fermion, which is the quark field in the QCD case, is,
\be
\Lag=\bar\psi\left(x\right)\left(i\gamma^\mu\partial_\mu\right)\psi(x)
\ee
where $\psi$ is the free-field for the fermion and $\gamma^\mu$ represents the Dirac matrices. The explanation below is valid with a mass term of $m\bar\psi\psi$ and $\cancel{\partial}_\mu \equiv \partial_\mu\gamma^\mu \equiv \partial^\mu\gamma_\mu$ is implied. The reason to discuss massless fermions is explained in the next section. The transformation of the fermion field under the $SU(3)_C$ group reads,
\be
\psi(x)\rightarrow e^{ig\frac{\lambda^a}{2}\theta_a(x)}\psi(x)
\ee
where $\frac{\lambda^a}{2}$ are the 8 Gell-Mann matrices and generators of the group. It is worth noting that $\partial_\mu\psi(x)$ does not transform in the same way and need a redefinition, such that,
\be
D_\mu=\partial_\mu-igA_\mu^a(x)\frac{\lambda^a}{2}
\ee
where the gauge vector fields $A_\mu^a$ represent the eight gluons. These gluon fields must transform in a specific way to satisfy the local gauge invariance, such that,
\be
A_\mu^a \rightarrow A_\mu^a+\partial_\mu\theta^a+gf^{abc}A_\mu^c\theta^c
\ee
where $f^{abc}$ denotes the structure constants according to the commutation $\left[ \frac{\lambda^a}{2}, \frac{\lambda^b}{2} \right] = if^{abc}\frac{\lambda^c}{2}$. We have achieved the invariance of the Lagrangian density by introducing the vector fields. Lastly, adding the kinetic term with the
\be
-\frac{1}{4}F_a^{\mu\nu}F_{\mu\nu}^a
\ee
form, where
\be
F_{\mu\nu}^a=\partial_\mu A_\nu^a-\partial_\nu A_\mu^a+gf^{abc}A_\mu^bA_\nu^c .
\ee
\label{qcd-kinetic-term}
The complete QCD Lagrangian density becomes,
\be
\Lag_{QCD} = \bar\psi\left(i\gamma^\mu\partial_\mu\right)\psi(x)-g\bar\psi(x)\gamma^\mu\frac{\lambda_a}{2}\psi(x)A_\mu^a-\frac{1}{4}F_a^{\mu\nu}F_{\mu\nu}^a
\ee
\label{qcd-lag}
where a summation convention is implied over all of the quark fields. In the Lagrangian density, the first term is the same free-field propagation as in the original equation and the second term emerges from the interaction of the quark field with the introduced vector field $A_\mu$ along with the covariant derivative. The constant $g$ parametrises the interaction strength and is usually redefined as $\alpha_s=g^2/4\pi$ which is the strong coupling constant. The third term in the \autoref{qcd-kinetic-term}, when subsitituted in \autoref{qcd-lag} in the kinetic term, results in the cubic and quartic self coupling of the gluon fields which are a general feature of the non-Abelian gauge theories.

The gluons and the quark fields are produced via imposing the local gauge invariance and $SU(3)_C$ group provides the 8 generators that are the 8 gluons of the strong force.

\subsection{Electroweak theory}

Electroweak interactions are explained with the same local gauge invariance but with the $SU(2)_L\otimes U(1)_Y$ group. The parity violating nature of the weak interactions requires defining different interactions of fermions of opposite chiralities. The left hand chirality projection operator is defined as $\frac{1-\gamma^5}{2}$ where the corresponding operator for the right hand chirality projection is $\frac{1+\gamma^5}{2}$ where $\gamma^5$ is defined as
\be
\gamma^5\equiv\gamma^0\gamma^1\gamma^2\gamma^3
\ee
For a massless particle, the chirality corresponds to the gelicity of that particle which is the normalised projection of the spin vector on the momentum vector.

The first part of the symmetry groups, the $SU(2)_L$ is associated with the weak isospin, actually the third component of it ($I_3$), and is a non-Abelian gauge group. The gauge invariance under the $SU(2)_L$ group introduces $W_\mu^i$, the gauge fields where i runs over the three generators of this group.. Fermion fields are stated with chirality components as left-handed doublets and right-handed singlets, and right handed particles do not interact with the $W_\mu^i$ gauge fields.

The second part, $U(1)_Y$ is an Abelian group. $B_\mu$ is the gauge field for the this group associated with weak hypercharge ($Y$) and results from the local gauge invariance. This field interacts with both the left and right handed particles. Here, the weak hypercharge symmetry is defined different than the Quantum Electrodynamics (QED) in order to unify the electrodynamics with the weak interactions. The relation between the electric charge Q, and the hypercharge $Y_W$ is given by the Gell-Mann-Nishjima formula;
\be
Q = I_3 + \frac{1}{2}Y_W
\label{hypercharge}
\ee
where $I_3$ is the third component of the weak isospin. The fermion fields, leptons in the case of weak interactions, can now be written as one doublet $\psi_L$ and two singlets $\psi_R$, $\psi\prime_R$, such that,
\be
\psi_L = \frac{1-\gamma^5}{2}
\begin{pmatrix}
    \psi \\
    \psi\prime
\end{pmatrix}
= \begin{pmatrix}
    \psi_L \\
    \psi\prime_L
\end{pmatrix} ,
\psi_R=\frac{1+\gamma^5}{2}
\begin{pmatrix}
    \psi \\
    \psi\prime
\end{pmatrix}
= \begin{pmatrix}
    \psi_R \\
    \psi\prime_R
\end{pmatrix}
\ee
where $\psi$ and $\psi\prime$ denotes either neutrino, charged lepton fields or up, down-type quark fields. A leptoquark coupling is not predicted in the SM nor observed, meaning that these two sectors are seperate. In other words, these two sectors cannot transform lepton fields into quark fields, vice-versa.

The Lagrangian density so far, consists of 3 main parts,
\be
\begin{aligned}
\Lag &= \Lag_{kin} + \Lag_{CC} + \Lag_{NC}\\
 &= i\bar\psi_L\cancel{D}\psi_L+i\bar\psi_R\cancel{D}\psi_R+i\bar\psi\prime_R\cancel{D}\psi\prime_R
\end{aligned}
\ee
where,
\be
D_\mu = \partial_\mu-igW_\mu^iT_i-ig\prime\frac{Y_\psi}{2}B^\mu
\ee
and $T_i=\frac{\sigma_i}{2}$ which represents the Pauli matrices which are the generators of the $SU(2)_L$ group with eigenvalues that are actually the weak isospin values. The Lagrangian density can then be rewritten as,
\be
\Lag_{kin} = i\bar\psi_L\cancel{\partial}\psi_L+i\bar\psi_R\cancel{\partial}\psi_R+i\bar\psi\prime_R\cancel{\partial}\psi\prime_R
\ee

\be
\begin{aligned}
\Lag_{CC} &= gW_\mu^1\bar\psi_L\gamma^\mu\frac{\sigma_1}{2}\psi_L+gW_\mu^2\bar\psi_L\gamma^\mu\frac{\sigma_2}{2}\psi_L\\
 &= \frac{g}{\sqrt{2}}W_\mu^+\bar\psi_L\gamma^\mu\sigma^+\psi_L+\frac{g}{\sqrt{2}}W_\mu^-\bar\psi_L\gamma^\mu\sigma^-\psi_L
\end{aligned}
\ee

\be
\begin{aligned}
\Lag_{NC} &= \frac{g}{\sqrt{2}}W_\mu^3
\left(\bar\psi_L\gamma_\mu\psi_L
-\bar\psi\prime_L\gamma_\mu\psi\prime_L\right)
+\frac{g\prime}{2}B_\mu
[Y_{\psi_L}\left(\bar\psi_L\gamma^\mu\psi_L
+\bar\psi\prime_L\gamma^\mu\psi\prime_L\right)\\
& +Y_{\psi_R}\left(\bar\psi_R\gamma^\mu\psi_R\right)Y_{\psi\prime_L}\left(\bar\psi\prime_R\gamma^\mu\psi\prime_R\right)]
\end{aligned}
\label{NClag}
\ee
where
\be
W_\mu^\pm = \frac{1}{\sqrt{2}}\left(W_\mu^1\mp iW_\mu^2\right)\sigma_\mu^\pm = \frac{1}{2}\left(\sigma^1\pm i\sigma^2\right)
\ee
which is the charged current interaction between the $\psi_L$ and $\psi\prime_L$ fields mediated by charged weak boson with $W_\mu^\pm$ fields. The neutral current field is described by the neutral Z boson field $Z_\mu$ and the photon field $A_\mu$. This interactions could not be intermediated by the $W_\mu^3$ field nor by $B_\mu$, since both couples to neutral fields. The relation between these fields are given by,
\be
B_\mu = A_\mu cos\theta_W-Z_\mu sin\theta_W
\label{Z1}
\ee
\be
W_\mu^3 = A_\mu sin\theta_W + Z_\mu cos\theta_W
\label{Z2}
\ee
where $\theta_W$ is the Weinberg angle. The neutral current interactions are derived by substituting \autoref{Z1} and \autoref{Z2} in \autoref{NClag}, resulting in the coupling strength of $gsin\theta_WI_3+g\prime cos\theta_W\frac{Y}{2}$ for $A_\mu$  field. The electroweak unification is achieved by equating this coupling strength to the photons field's coupling constant. Since the hypercharge is only multiplied by $g\prime$, we can set an arbitrary value for $Y_{\psi_L} = -1$. This process then gives, along with $Q = -1$ for leptons and 0 for neutrinos,
\be
gsin\theta_W = g\prime cos\theta_W = e
\ee
The Lagrangian density for the electroweak interactions then becomes,
\be
\Lag_{EW} = i\bar\psi_L\cancel{D}\psi_L+i\bar\psi_R\cancel{D}\psi_R+\bar\psi\prime_R\cancel{D}\psi\prime_R-\frac{1}{4}B_{\mu\nu}B^{\mu\nu}-\frac{1}{4}W_{\mu\nu}^iW_i^{\mu\nu}
\label{EWLag}
\ee
where
\be
B_{\mu\nu} = \partial_\mu B_\nu-\partial_\nu B_\mu
\ee
\be
W_{\mu\nu}^i = \partial_\mu W_{\nu}^i-\partial_\nu W_{\mu}^i + g\epsilon^{abc}W_\mu^bW_\nu^c
\ee

\autoref{EWLag} includes the charged and neutral interactions of fermions. Expanding the kinetic terms of the weak bosonic field yields the self couplings of the gauge bosons including the trilinear and quartic couplings. The mass terms for the bosonic fields violates the local gauge invariance as the same is valid for the strong force. Also, they are not allowed for the fermionic fields since the left and right chiralities transform differently which we will see in the next section. A list of fermionic fields is shown in \autoref{fermionfieldstable}.

\begin{table*}[ht]
	{\setlength{\tabcolsep}{14pt}
		\caption{Fermion fields under $SU(2)_L$ group representation.}
		\begin{center}
			\vspace{-6mm}
			\scalebox{0.78}{
			\begin{tabular}{cccccccc}
				\hline \\[-2.45ex] \hline \\[-2.1ex]
				Type&1^{st} gen.&2^{nd} gen.&3^{rd} gen.&I_3&Y&Q&SU(3)_C\\
				\hline \\[-1.8ex]
				 & \begin{pmatrix} u_L \\ d_L \end{pmatrix}&\begin{pmatrix} c_L \\ s_L \end{pmatrix}&\begin{pmatrix} t_L \\ b_L \end{pmatrix}&\begin{pmatrix} 1/2 \\ -1/2 \end{pmatrix}&1/3&\begin{pmatrix} 2/3 \\ -1/3 \end{pmatrix}&\\
				 Quarks&u_R&c_R&t_R&0&4/3&2/3&triplet\\
				 &d_R&s_R&b_R&0&-2/3&-1/3&\\
                \hline \\[-1.8ex]
				 & \begin{pmatrix} \nu_{e,L} \\ e_L \end{pmatrix}&\begin{pmatrix} \nu_{\mu,L} \\ \mu_L \end{pmatrix}&\begin{pmatrix} \nu_{\tau,L} \\ \tau_L \end{pmatrix}&\begin{pmatrix} 1/2 \\ -1/2 \end{pmatrix}&-1&\begin{pmatrix} 0 \\ -1 \end{pmatrix}&\\
				 Leptons&e_R&\mu_R&\tau_R&0&-2&1&singlet\\
				 &\nu_{e,R}&\nu_{\mu,R}&\nu_{\tau,R}&0&0&0&\\
				\hline
			\end{tabular}}
			\vspace{-6mm}
		\end{center}
		\label{fermionfieldstable}}
\end{table*}

The fermion fields share the same composition; the left and right chirality fields are doublet and singlets under the $SU(2)_L$ group, respectively, resulting in the possession of the weak isospin in the doublets. This interaction is mediated by the $W^\pm$ bosons where neutral interactions is mediated by the Z boson; it interacts with both chiralities but this time with a different coupling strength due to the mixing of the gauge fields shown in \autoref{Z1} and \autoref{Z2}. The electromagnetic interaction is mediated by photons that is only sensitive to the electric charge Q, hence to the hypercharge Y and the weak isospin $I_3$ with the relation given in \autoref{hypercharge}.

The interactions between fields may differ the quantum numbers by carrying charges; the strong interactions can change the colour charge of the quarks, and charged weak interactions can change the weak isospin, hence the electric charge of fermions.

The SM concept; explaining the matter fields by quantum numbers and the interactions by symmetry applications is an elegant mathematical theory, however the experimentally observed massive $W^\pm$ and Z bosons requires an addition to the theory, explained in the next section.

\subsection{Higgs mechanism}\label{higgsmechanismsection}

So far, we have built the SM on the assumption that the interactions are gauge invariant. This requires the vector bosons $W^\pm$ and $Z$ to be massless. In addition, for a fermion field $\psi$ satisfying the Dirac equation $ (i\hbar\gamma^\mu\partial_\mu-mc)\psi = 0$, the mass term $-m\bar\psi\psi$ arises which is actually not invariant under the $SU(2)_L$ gauge symmetry. This can be seen by expanding the mass term with left and right handed fermion fields, that is
\be
-m\bar\psi\psi = -m\left(\bar\psi_L\psi_R + \bar\psi_R\psi_L\right) ,
\ee
and have seen that the left-handed fields are doublets under $SU(2)_L$ while right-handed fields are singlet. This means that their gauge quantum numbers are different and this kind of mass term is forbidden. 
The solution to these theoretically massless but experimentally massive particles is given by \emph{the Brout-Englert-Higgs (BEH) Mechanism}~\cite{Higgs1964, BroutEnglert, Guralnik1964}, usually called \emph{the Higgs Mechanism}, showing that the electroweak symmetry can be broken spontaneously under specific conditions. 

The solution starts with introducing a complex scalar field $\phi$ to the theory, which is a doublet under $SU_L(3)$,
\be
 \phi = 
 \begin{pmatrix}
  \phi_+ \\
  \phi_0
 \end{pmatrix} ,
\ee\\
This field has the hypercharge value of $Y=1$ and the corresponding covariant derivative,
\be
D_\mu = \partial_\mu-igW_\mu^i\frac{\sigma_i}{2}-\frac{1}{2}ig\prime B_\mu .
\label{higgscovariantD}
\ee
The Lagrangian describing the Higgs field becomes,
\be
 \Lag_{Higgs} = \left(D_\mu\phi\right)^\dagger\left(D^\mu\phi\right)-V\left(\phi^\dagger\phi\right) .
 \label{HiggsLag}
\ee

Since the potential $V$ must satisfy SU(2) and U(1) symmetries, a good solution is a Mexican hat;
\be
 V(\phi) = -\mu^2\phi^\dagger\phi + \lambda\left(\phi^\dagger\phi\right)^2
 \label{higgspotential}
\ee
A spontaneous symmetry breaking (SSB) happens when both the constants $\mu^2$ and $\lambda$ are positive values. A convenient choice for the minimum would be
\be
\langle\phi\rangle=\frac{1}{\sqrt{2}}
 \begin{pmatrix}
  0 \\
  v
 \end{pmatrix} .
\ee
The minimum of this potential is non-zero with the potential shape in \autoref{mexicanHiggs} and given by,
\be
 \begin{aligned}
 \frac{dV}{d\left(\phi^\dagger\phi\right)} &=\mu^2+2\lambda\phi^\dagger\phi = 0 \\
 & \Rightarrow |\phi_{min}| = \sqrt{\frac{\mu^2}{2\lambda}} \equiv \frac{v^2}{2},
 \end{aligned}
\ee
where $v$ is called \emph{the vacuum expectation value} which is measured experimentally and found to be 246 GeV. The symmetry is broken once a particular ground state is chosen but the Lagrangian density remains gauge invariant. 
\vspace{6pt}
\begin{figure}[ht]
	\centering
	\includegraphics[width=\textwidth]{mexicanhat.png}
	\vspace{6pt}
	\caption{The Higgs potential V in the case that $\mu^2 > 0$. Choosing any point at the bottom of the potential breaks spontaneously the rotational U(1) symmetry.}
	\label{mexicanHiggs}
\end{figure}
Now let us expand the complex doublet field around the minimum $\phi(x)$,
\be
\langle\phi\rangle=\frac{1}{\sqrt{2}}e^{\frac{i\sigma_i\theta^i(x)}{v}}
 \begin{pmatrix}
  0 \\
  v+h(x)
 \end{pmatrix} .
 \label{higgsVparametrized}
\ee
This corresponds to a real scalar massive field, to three massless Goldstone boson fields $\theta^i$, which are not observed in nature. These  bosons may be created as many as the broken generators and can be removed with an $SU(2)_L$ transformation with a unitary gauge choice:
\be
\phi(x)\rightarrow\phi\prime(x)=e^{-\frac{i\sigma_i\theta^i(x)}{v}}\phi(x)=\frac{1}{\sqrt{2}}
 \begin{pmatrix}
    0 \\
    v+h(x)
 \end{pmatrix} .
 \label{Higgsunitarygauge}
\ee
This transformation yields the sole real scalar field. Substituting \autoref{higgscovariantD} and \autoref{Higgsunitarygauge}, the Lagrangian density with the Higgs field $h$, becomes
\be
\begin{aligned}
\Lag_{Higgs} &= \frac{1}{2}\partial^\mu h\partial_\mu h-\frac{1}{2}\left(2\lambda v^2\right)h^2\\
 & + \left[\left(\frac{gv}{2}\right)^2W^{\mu +}W_\mu^- +\frac{1}{2}\frac{\left(g^2+g\prime^2\right)v^2}{4}Z^\mu Z_\mu \right]\left(1+\frac{h}{v}\right)^2\\
 & -\lambda vh^3-\frac{\lambda}{4}h^4+\frac{\lambda}{4}v^4
\end{aligned}
\ee

The first two terms represent the kinetics of the Higgs field with the mass value $m_H^2 = 2\lambda v^2=2\mu^2$. The third term represents the weak boson masses with these values;
\be
m_W^2=\frac{g^2v^2}{4} \Rightarrow m_W = \frac{gv}{2} ,
\ee
\be
m_Z^2 = \frac{\left(g^2+g\prime^2\right)v^2}{4}=\frac{m_W^2}{cos^2\theta_W} \Rightarrow m_Z = \frac{m_W}{cos\theta_W}
\ee
It can bee seen that the removed Goldstone bosons returned as extra degrees of freedom for the $W^\pm$ and $Z$ bosons corresponding to their longitudinal polarisations. The weak bosons gained mass via this mechanism.  The third term of the Lagrangian defines the interactions between the Higgs and the weak bosons. There can be seen from the contributions that H-WW/ZZ interactions from $2H/v$ term and HH-WW/ZZ interactions from $H^2/v^2$ term. The final term demonstrates the cubic and quartic self-couplings of the Higgs field.
The potential now can be rewritten with the self-couplings as,
\be
V(h) = \frac{1}{2}m_h^2h^2+\lambda_{hhh}vh^3+\frac{1}{4}\lambda_{4h}h^4-\frac{\lambda}{4}v^4
\ee
where the self interaction coupling are defined as,
\be
\lambda_{hhh}=\lambda_{4h}=\frac{m_h^2}{2v^2}
\ee

Here, it is worth noting that these two self-couplings are solely dependent on the Higgs boson mass and the vacuum expectation value and their experimental measurement is a handy test for the SM and the electroweak symmetry breaking. There is one last term in the Higgs potential, $ -\lambda v^4/4$. This constant, being irrelevant in the SM, contributes to the cosmological constant and is not compatible with the astronomical observations. This is another mystery to be solved in the SM.
There is now the mass of the Higgs boson and the vacuum expectation value as free parameters to SM. The vacuum expectation value can be calculated precisely with the muon lifetime, such that,
\be
\frac{G_F}{\sqrt{2}} = \left(\frac{g}{2\sqrt{2}}\right)^2\frac{1}{m_W^2} \Rightarrow v = \sqrt{\frac{1}{\sqrt{2}G_F}} \approx 246 GeV
\ee
where $G_F$ is the Fermi constant.
Mass terms for the fermions are generated by the Higgs field's Yukawa interaction between the left and right chiral fields. The interaction's Lagrangian density can be written as,
\be
\Lag_{Yukawa} = -y_{f\prime}\left(\bar\psi_L\phi\psi\prime_R+\bar\psi\prime_R\phi^\dagger\psi_L\right)
- y_f\left(\bar\psi_L\tilde\phi\psi_R+\bar\psi_R\tilde\phi^\dagger\psi_L\right)
\ee
with the symmetry broken,
\be
\tilde\phi = i\sigma_2\phi^* = \begin{pmatrix} \phi_0^* \\ -\phi_+^* \end{pmatrix} \xRightarrow[\text{}]{\text{EWSB}} \frac{1}{\sqrt{2}} \begin{pmatrix} v+h(x)\\ 0 \end{pmatrix}
\ee
where $\psi$ denotes up and $\psi\prime$ denotes down-type fermions. When EWSB applied, the Lagrangian density reads,
\be
\Lag_{Yukawa} = -\sum_f m_f\left(\bar\psi_L\psi_R+\bar\psi_R\psi_L\right)\left(1+\frac{h}{v}\right)
\ee
where up and down-type fermions are summed with the mass terms,
\be
m_f\prime = y_f\prime\frac{v}{\sqrt{2}}
\ee
and the explanation for the fermion masses has been given by the interaction with the Higgs field. The strengths of the interactions depend on the fermion masses meaning that the massive the fermion, the stronger the interaction is.

Finally the Standard Model Lagrangian becomes,
\be
\begin{aligned}
\Lag_{SM} &= -\frac{1}{4}F_{\mu\nu}F^{\mu\nu}-\frac{1}{4}W_{\mu\nu}W^{\mu\nu}-\frac{1}{4}B_{\mu\nu}B^{\mu\nu}\\
 & +\bar\psi i\gamma^\lambda D_\lambda \psi + \left(D_\mu\phi\right)^\dagger\left(D^\mu\phi\right)-V\left(\phi^\dagger\phi\right)\\
 & + \Lag_{Yukawa}+ h.c.
\end{aligned}
\ee
and the EWSB is shown to be a renormalisable theory 't Hooft and Veltman \cite{thooft}.

The Higgs boson, required for the spontaneous symmetry breaking, was found experimentally by the CMS and ATLAS experiments at the LHC Experiment at CERN\cite{HiggsCMS,HiggsATLAS} in 2012, almost 50 years after its theoretical assumption. The mass of the Higgs boson is 125 GeV and its parameters; spin, parity and branching ratios are found to be consistent with the Standard Model predictions\cite{Higgsprecision1, Higgsprecision2}.

The most general form of the SM Lagrangian depends on 19 parameters. These parameters are given in \autoref{SMparameters}.
\begin{table*}[ht]
	{\setlength{\tabcolsep}{14pt}
		\caption{Parameters of the Standard Model.}
		\begin{center}
			\vspace{-6mm}
			\begin{tabular}{cccc}
				\hline \\[-2.45ex] \hline \\[-2.1ex]
				\# & Symbol & Name & Value \\
				\hline \\[-1.8ex]
				1 & $m_e$ & Electron mass & 0.511 MeV \\
				2 & $m_\mu$ & Muon mass & 105.7 MeV \\
				3 & $m_\tau$ & Tau mass & 1.78 GeV \\
				4 & $m_u$ & Up quark mass & 1.9 MeV \\
				5 & $m_d$ & Down quark mass & 4.4 MeV \\
				6 & $m_s$ & Strange quark mass & 87 MeV \\
				7 & $m_c$ & Charm quark mass & 1.32 GeV \\
				8 & $m_b$ & Bottom quark mass & 4.24 GeV \\
				9 & $m_t$ & Top quark mass & 173.5 GeV \\
				10 & $\theta_{12}$ & CKM 1-2 Mixing angle & 13.1\textdegree \\
				11 & $\theta_{23}$ & CKM 2-3 Mixing angle & 2.4\textdegree \\
				12 & $\theta_{13}$ & CKM 1-3 Mixing angle & 0.2\textdegree \\
				13 & $\delta$ & CKM CP violation Phase & 0.995 \\
				14 & $g_1$ or $g\prime$ & U(1) gauge coupling & 0.357 \\
				15 & $g_2$ or g & SU(2) gauge coupling & 0.652 \\
				16 & $g_3$ or $g_s$ & SU(3) gauge coupling & 1.221 \\
				17 & $\theta_{QCD}$ & QCD vacuum angle & $\sim $ 0 \\
				18 & $v$ & Higgs vacuum expectation value & 246 GeV \\
				19 & $m_H$ & Higgs mass & 125 GeV \\
				\hline
			\end{tabular}
			\vspace{-6mm}
		\end{center}
		\label{SMparameters}}
\end{table*}

\section{The Higgs Boson Phenomenology and Experimental Status}

The Higgs boson, with its zero spin, is different from any other particle in the Standard Model. It does not carry any electric or colour charge hence does not take part in the electromagnetic and strong interactions. Nonetheless, it interacts with fermions and heavy bosons including itself, since it carries weak isospin. The coupling strengths of the Higgs boson to those particles are derived in \autoref{higgsmechanismsection}. These couplings are solely dependent on the masses of the particles which Higgs couple to, resulting in stronger coupling strengths to heavier particles\cite{pdg}.

The discovery of the Higgs Boson in 2012 was made with the data collected at the Run-1 of the CMS and ATLAS Detectors at $\sqrt{s} = 7$ and $8$ TeV with the integrated luminosity of about $5 fb^{-1}$ in its decays to $H\rightarrow bb$, $H\rightarrow ZZ$, $H\rightarrow \gamma\gamma$, $H\rightarrow WW$ and $H\rightarrow \tau\tau$. Further data at $\sqrt{s} = 13$ TeV collected with the same detectors at Run-2 confirmed the discovery, shown in \autoref{higgs-13tev}. The mass of the Higgs boson is given by the Particle Data Group\cite{pdg} as,
\be
m_H = 125.25\pm0.17 ,
\ee
and a summary of the mass measurements resulting to this value is shown in \autoref{higgsmasssummary}.

\begin{figure}[ht]
	\centering
	\includegraphics[width=\textwidth]{higgsmasscombined.png}
	\caption[Summary of the CMS and ATLAS Higgs mass measurements in the $\gamma\gamma$ and $ZZ$ channels in Run I and Run II.]{Summary of the CMS and ATLAS Higgs mass measurements in the $\gamma\gamma$ and $ZZ$ channels in Run I and Run II\cite{pdg}.}
	\label{higgsmasssummary}
\end{figure}

The Higgs boson was also shown to have a spin-parity, $J^P=0^+$ \cite{higgs-spin}. The results of the Higgs boson searches showed good agreement with the Standard Model predictions. 

\begin{figure*}[ht]
        \centering
        \begin{subfigure}[b]{0.475\textwidth}
            \centering
            \includegraphics[width=\textwidth]{MSc_Thesis/fig/cms-higgs-13tev.png}
            \firstsubcaption{$m_{4l}$}
        \end{subfigure}
        \hspace{0.2cm}
        \begin{subfigure}[b]{0.475\textwidth}  
            \centering 
            \includegraphics[width=\textwidth]{MSc_Thesis/fig/atlas-higgs-13tev.png}
            \firstsubcaption{$m_{\gamma\gamma}$}
        \end{subfigure}
        \caption[Four-lepton mass distribution, $m_{4l}$ with 2 GeV bin size obtained from the data collected at the CMS Detector on the left, and diphoton invariant mass distribution obtained from the data collected at the ATLAS Detector on the right, both at $\sqrt{s}=13$ TeV in Run II.]
        {\small Four-lepton mass distribution, $m_{4l}$ with 2 GeV bin size obtained from the data collected at the CMS Detector\cite{cms-higgs-13tev} on the left, and diphoton invariant mass distribution obtained from the data collected at the ATLAS Detector\cite{atlas-higgs-13tev} on the right, both at $\sqrt{s}=13$ TeV in Run II.} 
        \label{higgs-13tev}
\end{figure*}

The phenomenology of the Higgs boson at the LHC (the LHC is explained in \autoref{Ch2}), has been thoroughly studied \cite{higg-phen-1,higg-phen-2,higg-phen-3}. Four main production modes of the Higgs boson at the hadron colliders have been discovered. Their Feynman diagrams are shown in \autoref{HiggsFeynman}. The production modes other than these four have very small cross sections.

\begin{figure*}[ht]
        \centering
        \begin{subfigure}[b]{0.475\textwidth}
            \centering
            \includegraphics[width=\textwidth]{GGFH.png}
            \firstsubcaption{$ggF$}
            \label{GGFH}
        \end{subfigure}
        \hfill
        \begin{subfigure}[b]{0.475\textwidth}  
            \centering 
            \includegraphics[width=\textwidth]{VBFH.png}
            \firstsubcaption{$VBF$}  
            \label{VBFH}
        \end{subfigure}
        \vskip\baselineskip
        \begin{subfigure}[b]{0.475\textwidth}   
            \centering 
            \includegraphics[width=\textwidth]{VH.png}
            \firstsubcaption{$VH$}    
            \label{VH}
        \end{subfigure}
        \hfill
        \begin{subfigure}[b]{0.475\textwidth}   
            \centering 
            \includegraphics[width=\textwidth]{ttH.png}
            \firstsubcaption{$t\bar tH$}   
            \label{ttH}
        \end{subfigure}
        \caption[ Feynman diagrams of Higgs boson productions at the LHC. The gluon fusion ($ggF$), vector boson fusion ($VBF$), Higgs strahlung ($VH$) and associated production with top quarks ($t\bar tH$) are shown. In the diagrams, $q$ denotes any quark, $t$ denotes the top quark, $V$ denotes any of the $Z$ or $W^\pm$ bosons.]
        {\small Feynman diagrams of Higgs boson productions at the LHC. The gluon fusion ($ggF$), vector boson fusion ($VBF$), Higgs strahlung ($VH$) and associated production with top quarks ($t\bar tH$) are shown. In the diagrams, $q$ denotes any quark, $t$ denotes the top quark, $V$ denotes any of the $Z$ or $W^\pm$ bosons.} 
        \label{HiggsFeynman}
    \end{figure*}

The cross sections of Higgs productions at the proton-proton (pp) collisions at the LHC is shown in \autoref{HiggsxsecTable}.

\begin{table*}[ht]
	{\setlength{\tabcolsep}{14pt}
		\caption{Cross section (in pb) of the Higgs boson production at different centre-of-mass ($\sqrt{s}$) energies for $m_H=125$ GeV.}
		\begin{center}
			\vspace{-6mm}
			\begin{tabular}{cccccc}
				\hline \\[-2.45ex] \hline \\[-2.1ex]
				$\sqrt{s}$ (TeV) &&&Production mode&&\\
				\hline \\[-1.8ex]
				& $ggF$ & $VBF$ & $VH$ & $t\bar tH$ & total \\
				\hline \\[-1.8ex]
                1.96 & 0.95 & 0.065 & 0.209 & 0.004 & 1.23 \\
                7 & 16.9 & 1.24 & 0.92 & 0.09 & 19.1 \\
                8 & 21.4 & 1.60 & 1.12 & 0.13 & 24.2 \\
                13 & 48.6 & 3.78 & 2.25 & 0.5 & 55.1 \\
                14 & 54.7 & 4.28 & 2.5 & 0.61 & 62.1 \\
				\hline
			\end{tabular}
			\vspace{-6mm}
		\end{center}
		\label{HiggsxsecTable}}
\end{table*}

In \emph{the gluon fusion production} mode, shown at \autoref{GGFH}, two gluons produce a Higgs boson via a fermion loop but this fermion is usually the top quark since the Higgs boson couples to massive particles more often. This production mode has a cross section of 48.6 pb at $\sqrt{13}$ TeV and at least one order of magnitude larger than the second likely production mode at the pp collions. The Higgs boson produced in this mode in leading order, is expected to have a small transverse momentum which can ease the event selection to seperate the Higgs signal from background processes. 

\emph{The vector boson fusion production} mode, with a cross section of 3.78 pb at $\sqrt{13}$ TeV, is shown in \autoref{VBFH}. This production mode scatters the quarks out of the protons and these quarks help characterising this production mode with two additional jets in the final state. This process can be observed in the particle detectors as two back-to-back jets that carry high transverse momentum with high absolute pseudo-rapidities.

WH and ZH associated production mechanism of the Higgs boson, or \emph{Higgs strahlung}, is the process of producing a $W^\pm$ or Z boson where a Higgs boson is radiated away. This production mode has a cross section of 2.25 pb at $\sqrt{13}$ TeV. The vector boson in the final state is a good signature for signal and background separation when decayed leptonically.

The fourth and final production mode that we look at is the \emph{associated production with top quark pair}. This production mechanism is a rare process in pp collisions with a cross section of 0.5 pb at $\sqrt{13}$ TeV. This production mechanism is observed at the CMS Experiment in 2018 \cite{ttH-higgs}. The final state objects in this production mode, the top pair and the Higgs boson, can be identified by the large multiplicity of jets and leptons, however due to the lower cross section and the complex final state objects, this production mode is usually not included in the analyses and the same is valid for this thesis. When all of the standard model processes considered, the total cross section for the Higgs boson production makes it a rare process.

The cross sections of these production modes as a function of $\sqrt{s}$ is shown in \autoref{higgs-xsec-vs-sqrtS}.

\begin{figure*}[ht]
        \centering
        \begin{subfigure}[b]{0.475\textwidth}
            \centering
            \includegraphics[width=\textwidth]{MSc_Thesis/fig/higgs-xsec-vs-sqrtS.png}
            \firstsubcaption{$\sigma_H$}
            \label{higgs-xsec-vs-sqrtS}
        \end{subfigure}
        \hspace{0.2cm}
        \begin{subfigure}[b]{0.475\textwidth}  
            \centering 
            \includegraphics[width=\textwidth]{MSc_Thesis/fig/higgs-br-vs-mH.png}
            \firstsubcaption{$m_H$}
            \label{higgs-br-vs-mH}
        \end{subfigure}
        \caption[Higgs boson production cross section as a function of different production mechanisms (on the left). Branching fractions the Higgs boson as a function of $m_H$ .The theoretical uncertainties are indicated as bands in both plots.]
        {\small Higgs boson production cross section as a function of different production mechanisms\cite{higg-phen-3} (on the left). Branching fractions the Higgs boson as a function of $m_H$\cite{higgs-br-vs-mH}.The theoretical uncertainties are indicated as bands in both plots.}
\end{figure*}

The decay modes of the standard model Higgs boson around its mass is shown in \autoref{higgs-br-vs-mH}, and branching fractions are given in \autoref{HiggsBRtable}. The decay of Higgs boson to fermion has been first established in the $\tau^+\tau^-$ decay mode\cite{CMS-PAS-HIG-16-043}, further confirming the combined results from the ATLAS and CMS Experiments \cite{Aad2016}. In the recent years, the decay of the Higgs boson to a muon pair has been studied and an excess of events is observed in data corresponding to an integrated luminosity of $137 fb^{-1}$ with a significance of 3.0 standard deviations\cite{CMS-PAS-HIG-19-006}, see in \autoref{htomumu.png}.

\begin{table*}[ht]
	{\setlength{\tabcolsep}{14pt}
		\caption[The branching ratios and the relative uncertainty for a SM Higgs boson with $m_H$ = 125 GeV.]{The branching ratios and the relative uncertainty for a SM Higgs boson with $m_H$ = 125 GeV\cite{higgs-br-vs-mH, higg-phen-3}.}
		\begin{center}
			\vspace{-6mm}
			\begin{tabular}{lcr}
				\hline \\[-2.45ex] \hline \\[-2.1ex]
				Decay channel & Branching fraction & Rel. Uncertainty \\
				\hline \\[-1.8ex]
                $H\rightarrow \gamma\gamma$ & $2.27x10^{-3}$ & $2.1\%$\\
                $H\rightarrow ZZ$ & $2.62x10^{-2}$ & $\pm 1.5\%$\\
                $H\rightarrow W^+W^-$ & $2.14x10^{-1}$ &$\pm 1.5\%$ \\
                $H\rightarrow \tau^+\tau^-$ & $6.27x10^{-2}$ &$\pm 1.6\%$ \\
                $H\rightarrow b\bar b$ & $5.82x10^{-1}$ & $^{+1.2\%}_{-1.3\%}$\\
                $H\rightarrow c \bar c$ & $2.89x10^{-2}$ &$^{+5.5\%}_{-2.0\%}$ \\
                $H\rightarrow Z\gamma$ & $1.53x10^{-3}$ & $\pm 5.8\%$\\
                $H\rightarrow \mu^+\mu^-$ & $2.18x10^{-4}$ &$\pm 1.7\%$ \\
				\hline
			\end{tabular}
			\vspace{-6mm}
		\end{center}
		\label{HiggsBRtable}}
\end{table*}

Since the decay channels of the Higgs is strictly dependent on its mass, the branching ratios vary too much as can be seen in \autoref{higgs-br-vs-mH-complete}. At high Higgs masses, the decys into the $W^\pm$ and $Z$ bosons dominate the spectrum. On the other hand, the decays in the lower mass region where $m_H \le 200\; GeV$ ... 


\begin{figure}[ht]
	\centering
	\includegraphics[width=\textwidth]{higgs-br-vs-mH-complete.png}
	\caption[Branching ratio of the Higgs boson decaying into various final states as a function of the Higgs boson mass.]{Branching ratio of the Higgs boson decaying into various final states as a function of the Higgs boson mass\cite{higg-phen-3}.}
	\label{higgs-br-vs-mH-complete}
\end{figure}









\section{BSM Searches}

