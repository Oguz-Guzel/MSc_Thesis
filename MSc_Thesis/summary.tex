Since the discovery of a Higgs boson in 2012 by the CMS and ATLAS experiments at the CERN's Large Hadron Collider in Geneva, Switzerland, physicists have tried to measure accurately its properties and to understand better the underlying electroweak symmetry breaking mechanism. In this pursuit, the search for Higgs boson pair production is crucial to test our understanding of the Higgs potential and to search for clues for the Beyond the Standard Model searches. This thesis describes the search for the Higgs boson pair production in decays to a W boson pair and a photon pair. The \ttgg channel of the Higgs boson pair decay is analysed alongside since an overlap is expected in the final states. Monte Carlo simulations of proton-proton collisions corresponding to an integrated luminosity of 3000 $fb^{-1}$ at a centre-of-mass energy of 14 TeV are used. The gluon-gluon fusion production mode of the Higgs boson pair is considered only. The \textsc{Delphes} fast detector simulation is used with an average pile-up of 200 per interaction with a dedicated card for Phase-2-upgraded CMS detector. A Python-based analysis library called \textsc{Bamboo} is used to perform the object selections and event categorisation in the data analysis of the study. Cut-flow tables reporting the number of events at each final state of interest are shown. Two multi-class Deep Neural Networks (DNN) are employed using the \textsc{Keras} API for \textsc{TensorFlow} machine learning library in order to increase the signal and background discrimination in the semi-leptonic final state of \wwgg and single $\tau$ final state of \ttgg channels. DNN score cuts are applied to each final state and the di-photon invariant mass distributions are obtained. The results are then used in the \textsc{Higgs Combine Tool} with the statistical and systematic uncertainties applied. The significance levels are reported for each final state along with their combination.