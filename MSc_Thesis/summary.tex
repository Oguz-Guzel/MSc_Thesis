Since the discovery of a Higgs boson in 2012 by the CMS and ATLAS experiments at the CERN's Large Hadron Collider, physicists have been trying to measure accurately its properties and to understand better the underlying electroweak symmetry breaking mechanism. In this pursuit, the search for Higgs boson pairs is crucial to test our understanding of the Higgs potential and to search for clues for the Beyond the Standard Model searches. This thesis describes the search for the Higgs boson pairs in decays to $\tau$ leptons and photon pairs. The \wwgg and \zzgg channels of the Higgs boson pairs are analysed alongside since an overlap is expected with the \ttgg channel. Monte Carlo simulations of proton-proton collisions corresponding to an integrated luminosity of 3000 $fb^{-1}$  at a centre-of-mass energy of 14 TeV are used. The gluon-gluon fusion and vector boson fusion production modes of Higgs boson pairs are considered. The Delphes Detector Simulation is used with average pile-up of 200 per interaction with a dedicated card for Phase-II CMS detector. A Python based analysis library called Bamboo is used for the object selections and event categorisation. Performance of object selections are computed and a cut-flow table reporting the number of events at each final state of interest is shown. A Deep Neural Network (DNN) is employed with the Keras class of TensorFlow in order to increase the signal and background discrimination in the study. A DNN score cut is applied to each final state and the distributions are obtained. These results are then used in the Higgs Combine Tool with the statistical and systematic uncertainties to extract the significance values and likelihood scans. The tri-linear Higgs self coupling parameters are measured in terms of coupling modifiers with respect to Standard Model in effective field theory of \kl framework. The results are seen in well agreement with the Standard Model expectations.