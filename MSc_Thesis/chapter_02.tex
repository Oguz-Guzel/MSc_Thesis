\chapter{THE LHC AND THE CMS EXPERIMENT}\label{Ch2}

European Organisation for Nuclear Research, best know as CERN, is established in 1954 by 12 European countries and is based at the Franco-Swiss border in northwest Geneva, Switzerland. Today, the organisation operates the largest particle physics laboratory in the world and has 23 member states \cite{CERN:2771424} and many others that make up more than 11 thousand researches and more than 70 countries around the globe. CERN's main activity is to provide the particle accelerators and needed infrastructure for high energy physics research. The organisation hosts the LHC Experiment which is the world's largest particle collider.

This chapter introduces the structure and operations of the LHC and the CMS Detector on which the simulated Monte Carlo event samples are based in this thesis.

\section{The Large Hadron Collider}

The Large Hadron Collider consists of a 27-kilometre ring tunnel where the beams of particles (protons or lead ions) are accelerated with the help of superconducting magnets along with a number of accelerating systems. It was built between 1998 and 2008 and it lies 175 metres beneath the surface. The initial design of the LHC aimed at the centre-of-mass energy of $\sqrt{s}=14$ TeV with a nominal peak luminosity of $\mathcal{L} = 10^{34}cm^{-2}s^{-1}$ \cite{Baconnier:257706}. The LHC collides proton-proton beams as well as lead-lead (Pb - Pb), proton-lead (p - PB) and Xenon-Xenon (Xe - Xe) nuclei to study heavy-ion collisions. The \textbf{\emph{LHC accelerator complex}} consist of a series of accelerators and is used to accelerate protons before being injected in the LHC, which will be explained shortly.

\subsection{Operation of the LHC}

The success of the LHC required two decade-long international collaboration. The realisation of the started in the early 1980s when the The Large Electron Positron Collider (LEP) at CERN was not even running. CERN Council approved the construction of the LHC in 1994 and a technical design report was published on year later. Starting from 1998, the construction of the LHC was completed after 10 years and in the same year it succeeded to accelerate proton beams \cite{lhcfirstbeams}. 