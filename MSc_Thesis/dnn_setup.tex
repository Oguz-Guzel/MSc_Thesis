\begin{lstlisting}[language=Python, caption=DNN setup for the \wwgg semi-leptonic final state, label={dnncode}]
import os
import yaml
import importlib
import matplotlib.pyplot as plt
from matplotlib.backends.backend_pdf import PdfPages
import numpy as np
import pyarrow.parquet as pq
import pandas as pd
from sklearn.model_selection import train_test_split
from sklearn.preprocessing import LabelEncoder, OneHotEncoder 
import tensorflow as tf
from tensorflow import keras
from tensorflow.keras import layers
from tensorflow.keras.callbacks import EarlyStopping, ReduceLROnPlateau
### Parameters of the training ###

#split = "even" 
split = "odd" 
# split = even | odd -> on what split to train the model (will be in the name)
# -> you need one "odd" and one "even" models to be put inside bamboo

suffix = 'test3'
# Suffix that will be added to the saved model (so multiple DNNs can be trained)

quantile = 1.0 # We will repeat the part of the weights rightmost tail
# Eg : 0.95, means we take the 5% events on the right tail of training 
# weight and repeat them
# 1.0 means no correction (to be used if you want to diable it)

tags = ['HH','background','single H']
importance = [1.,1.,1.] # Importance of each category

# DNN hyperparameters #
parameters = {
    'epochs'                : 200,
    'lr'                    : 0.001,
    'batch_size'            : 256,
    'n_layers'              : 3,
    'n_neurons'             : 64,
    'hidden_activation'     : 'relu',
    'output_activation'     : 'softmax',
    'l2'                    : 1e-6,
    'dropout'               : 0.,
    'batch_norm'            : True,
}
# L2 is an additional term in the loss function : l2 x ||W||**2 where 
# ||W|| is the sum of all the DNN weights 
#    inside the neurons
# -> when overfitting the weights take large values, this tells the 
# optimizer the trade off between performances
#    and generalization (from experience, a small value always helps)
# Dropout is a frequency of killing neurons at each batch
# (no backprogation for them)
# -> used generally when overfitting is detected, to avoid
# that the DNN learns too much 
#    (from experience, not always useful, put it when you see overfitting)
# Batch normalization is a layer that normalizes the output
#  of each neuron (see documentation)
# -> usually helps that the gradient does not go too far and
#  backprogation is always smooth (supposedly)
#    (from experience : always used it, maybe not worth all the time)


# Input variables
input_vars=["Eta_ph1",
            "Phi_ph1",
            "E_mGG_ph1",
            "pT_mGG_ph1",
            "Eta_ph2",
            "Phi_ph2",
            "E_mGG_ph2",
            "pT_mGG_ph2",
            "Electron_E",
            "Electron_pT",
            "Electron_Eta",
            "Electron_Phi",
            "Muon_E",
            "Muon_pT",
            "Muon_Eta",
            "Muon_Phi",
            "nJets",
            "E_jet1",
            "pT_jet1",
            "Eta_jet1",
            "Phi_jet1",
            "E_jet2",
            "pT_jet2",
            "Eta_jet2",
            "Phi_jet2",
            "InvM_jet",
            "InvM_jet2",
            "met"]
# Load the required data #
outputPath = '/home/ucl/cp3/sjain/bamboodev/WWGG/Clean_Skim'
skimFile = os.path.join(outputPath,'results','Skim.parquet')
yamlFile = os.path.join(outputPath,'plots.yml')

# Load dataframe from parquet #
df = pd.read_parquet(skimFile)

# Load samples + plots data from yaml file #
with open(yamlFile,'r') as handle:
    config = yaml.full_load(handle)

# Cut negative event weights #
#df = df[(df['weight']>0) & (df['weight']<300)]
df = df[df['weight']>0]

# Cut ttH events #
#df = df[~df.process.str.contains("ttHJetToGG")]

# Add tag column #
if 'tag' in df.columns:
    del df['tag']

df['tag'] = 'background'
for singleH in ['VBFH','VH','THQ','GluGluHTo','ttHJet']:
    df.loc[df.process.str.contains(singleH),['tag']] = 'single H'
df.loc[df.process.str.contains('HH'),['tag']] = 'HH'

for tag in tags:
    if tag in df.columns:
        del df[tag]

assert len(set(tags).intersection(set(pd.unique(df['tag'])))) == \
    len(tags) # Just cross check to avoid mistakes
# Add onehot column #
label_encoder  = LabelEncoder()                                                                                                                                                              
onehot_encoder = OneHotEncoder(sparse=False)
label_encoder.fit(tags)
integers = label_encoder.transform(df['tag']).reshape(-1, 1)
onehot_encoder.fit(np.arange(len(tags)).reshape(-1, 1))
onehot = onehot_encoder.transform(integers)
onehot_df = pd.DataFrame(onehot,columns=tags,index=df.index)

df = pd.concat((df,onehot_df),axis=1)

df['event_weight'] = df['weight'].copy()

if (df['event_weight'] < 0).sum() > 0:
    raise RuntimeError(f"There are {(df['event_weight'] < 0).sum()} \
        events with negative event weight, this should not happen")

# Compute training weight #
if 'training_weight' in df.columns:
    del df['training_weight']
df['training_weight'] = df['event_weight'].copy()

for itag,tag in enumerate(tags):
    df.loc[df['tag']==tag,'training_weight'] *= importance[itag] \
        * df.shape[0]/ len(tags) / df[df['tag']==tag]['event_weight'].sum()

df = df[(df['weight']>0) & (df['training_weight']<200)]

import utils
importlib.reload(utils) # Reload in case file has changed
print ('Using event weight')
utils.checkBatches(df,label_column='tag',weight_column='event_weight',\
    batch_size=parameters['batch_size'])
print ('Using training weight')
utils.checkBatches(df,label_column='tag',weight_column='training_weight',\
    batch_size=parameters['batch_size'])

# Plot the background and signal weights #
fig,axs = plt.subplots(figsize=(16,8),nrows=len(tags),ncols=3)
fig.subplots_adjust(left=0.1, right=0.9, top=0.98, bottom=0.1, wspace=0.2,\
    hspace=0.4)
for irow,tag in enumerate(tags):
    for icol,column in enumerate(['weight','event_weight','training_weight']):
        axs[irow,icol].hist(df[df['tag']==tag][column],bins=100,color='b')
        axs[irow,icol].set_title(f"Category = {tag}")
        axs[irow,icol].set_xlabel(column)
        axs[irow,icol].set_yscale('log')
fig.savefig("event_weights_A.pdf", dpi = 300)

# Determine splitting variable #
split_var = df['Phi_ph1'].copy()
split_var = np.abs(split_var)
split_var *= 1e4
split_var -= np.floor(split_var) 
split_var = (split_var*1e1).astype(int)
split_var = split_var %2 == 0
print (f'Even set has {df[split_var].shape[0]:10d} events \
    [{df[split_var].shape[0]/df.shape[0]*100:5.2f}%]')
print (f'Odd  set has {df[~split_var].shape[0]:10d} events \
    [{df[~split_var].shape[0]/df.shape[0]*100:5.2f}%]')

# Sets splitting #
print (f'Using split type {split}')
# split_var = True (even number) | False (odd number)
# Name of the model is related to the even 
# or odd quality of the events during inference (ie, in bamboo)
if split == 'even':
    train_df = df[~split_var] # Trained on odd
    test_df  = df[split_var]  # Evaluated on even 
elif split == 'odd':
    train_df = df[split_var]  # Trained on even
    test_df  = df[~split_var] # Evaluated on odd 
else:
    raise RuntimeError(f'Split needs to be either odd or even, is {split}')

# Randomize for training (always good to randomize) #
train_df = train_df.sample(frac=1)

# Quantile corrections #
# When an event has a large weight, it can imbalance a lot the training,
# still the weight might have a meaning
# Idea : instead of 1 event with wi>>1, we use N copies of the event
# with wf = wi/N
# From the point of view of the physics it does not matter, the total
# event weight sum of each process is the same
# From the point of view of the DNN, we have split a tough nut to crack
# into several smaller ones

quantile_lim = train_df['training_weight'].quantile(quantile)
print (f'{(1-quantile)*100:5.2f}% right quantile is when weight is at \
    {quantile_lim}')
print ('  -> These events will be repeated and their learning weights \
    reduced accordingly to avoid unstability') 

# Select the events #
idx_to_repeat = train_df['training_weight'] >= quantile_lim                          
events_excess = train_df[idx_to_repeat].copy()

saved_columns = train_df[['training_weight','process']].copy()

# Compute multiplicative factor #
factor = (events_excess['training_weight']/quantile_lim).values.astype(np.int32) 

# Correct the weights of events already in df #
train_df.loc[idx_to_repeat,'training_weight'] /= factor

# Add N-1 copies #
arr_to_repeat = train_df[idx_to_repeat].values                                       
repetition = np.repeat(np.arange(arr_to_repeat.shape[0]), factor-1)                   
df_repeated = pd.DataFrame(np.take(arr_to_repeat,repetition,axis=0),
columns=train_df.columns)
df_repeated = df_repeated.astype(train_df.dtypes.to_dict()) 
# otherwise dtypes are object
train_df = pd.concat((train_df,df_repeated),axis=0,ignore_index=True)\
    .sample(frac=1).reset_index() # Add and randomize

# Printout #
print ('Changes per process in training set')
for process in pd.unique(train_df['process']):
    N_before = saved_columns[saved_columns['process']==process].shape[0]
    N_after  = train_df[train_df['process']==process].shape[0]
    if N_before != N_after:
        print (f"{process:20s}")
        print (f"... {N_before:6d} events [sum weight = \
            {saved_columns[saved_columns['process']==process]\
                ['training_weight'].sum():14.6f}]",end=' -> ')
        print (f"{N_after:6d} events [sum weight = \
            {train_df[train_df['process']==process]\
                ['training_weight'].sum():14.6f}]")
    
print ()
print (f"Total entries : {saved_columns.shape[0]:14d} -> \
    {train_df.shape[0]:14d}")
print (f"Total event sum : {saved_columns['training_weight']\
    .sum():14.6f} -> {train_df['training_weight'].sum():14.6f}")

# Validation split #
train_df,val_df  = train_test_split(train_df,test_size=0.3)

# Printout #
print ('\nFinal sets')
print (f'Training set   = {train_df.shape[0]}')
print (f'Validation set = {val_df.shape[0]}')
print (f'Testing set    = {test_df.shape[0]}')
print (f'Total set      = {df.shape[0]}')

# Plot the background and signal weights #
fig,axs = plt.subplots(figsize=(16,8),nrows=1,ncols=2)
fig.subplots_adjust(left=0.1, right=0.9, top=0.96, bottom=0.1,
                    wspace=0.2,hspace=0.3)

if split == 'even':
    axs[0].hist(df[~split_var]['training_weight'],bins=100,color='b')
elif split == 'odd':
    axs[0].hist(df[split_var]['training_weight'],bins=100,color='b')
axs[0].set_title("Before correction")
axs[0].set_xlabel("Training weight")
axs[0].set_yscale('log')
axs[1].hist(train_df['training_weight'],bins=100,color='b')
axs[1].set_title("After correction")
axs[1].set_xlabel("Training weight")
axs[1].set_yscale('log')
fig.savefig("event_weights_C.pdf", dpi = 300)

# Input layer #
inputs = keras.Input(shape=(len(input_vars),), name="particles")

# Preprocessing layer
from tensorflow.keras.layers.experimental import preprocessing
normalizer = preprocessing.Normalization\
    (mean     = train_df[input_vars].mean(axis=0),
    variance = train_df[input_vars].var(axis=0),
    name     = 'Normalization')(inputs)
    # this layer does the preprocessing (x-mu)/std for each input
# Dense (hidden) layers #
x = normalizer
for i in range(parameters['n_layers']):
    x = layers.Dense(
        units                = parameters['n_neurons'], 
        activation           = parameters['hidden_activation'], 
        activity_regularizer = tf.keras.regularizers.l2(parameters['l2']),
        name                 = f"dense_{i}")(x)
    if parameters['batch_norm']:
        x = layers.BatchNormalization()(x)
    if parameters['dropout'] > 0.:
        x = layers.Dropout(parameters['dropout'])(x)
# Output layer #
outputs = layers.Dense(
    units                = 3, 
    activation           = parameters['output_activation'],
    activity_regularizer = tf.keras.regularizers.l2(parameters['l2']),
    name                 = "predictions")(x)

# Registering the model #
model = keras.Model(inputs=inputs, outputs=outputs)

model_preprocess = keras.Model(inputs=inputs, outputs=normalizer)
out_test = model_preprocess.predict(train_df[input_vars],batch_size=5000)
print ('Input (after normalization) mean (should be close to 0)')
print (out_test.mean(axis=0))
print ('Input (after normalization) variance (should be close to 1)')
print (out_test.var(axis=0))

model.compile(
    #optimizer=keras.optimizers.RMSprop(),
    optimizer=keras.optimizers.Adam(lr=parameters['lr']),  # Optimizer
    # Loss function to minimize
    loss=keras.losses.CategoricalCrossentropy(),
    # List of metrics to monitor
    metrics=[keras.metrics.BinaryAccuracy(),
             tf.keras.metrics.AUC(),
             tf.keras.metrics.Precision(),
             tf.keras.metrics.Recall()],
)

model.summary()

# Callbacks #
early_stopping = EarlyStopping(monitor = 'val_loss',
                               min_delta = 0.001, 
                               patience = 20,
                               verbose=1,
                               mode='min',
                               restore_best_weights=True)
# Stop the learning when val_loss stops increasing 
# https://keras.io/api/callbacks/early_stopping/

reduce_plateau = ReduceLROnPlateau(monitor = 'val_loss',
                                   factor = 0.1,
                                   min_delta = 0.001, 
                                   patience = 8,
                                   min_lr = 1e-8,
                                   verbose=2,
                                   mode='min')
# reduce LR if not improvement for some time 
# https://keras.io/api/callbacks/reduce_lr_on_plateau/
import History 
importlib.reload(History)
loss_history = History.LossHistory()

history = model.fit(
    train_df[input_vars],
    train_df[tags],
    verbose=2,
    batch_size=parameters['batch_size'],
    epochs=parameters['epochs'],
    sample_weight=train_df['training_weight'],
    # We pass some validation for
    # monitoring validation loss and metrics
    # at the end of each epoch
    validation_data=(val_df[input_vars],val_df[tags],
                    val_df['training_weight']),
    callbacks = [early_stopping, reduce_plateau, loss_history],
)

History.PlotHistory(loss_history,params=parameters,
                    outputName=f'loss_{suffix}_{split}.png')
# Params is a dict of parameters with name and values
# used for plotting

# Produce output on the test set as new column #
output = model.predict(test_df[input_vars],batch_size=5000)
output_tags = [f'output {tag}' for tag in tags]
    # Here the batch_size arg is independent of the learning
    # Default is 32, but it can become slow, by using large 
    # value it will just compute more values in parallel
    # (more or less parallel, we are not using a GPU)
for output_tag in output_tags:
    if output_tag in test_df.columns:
        # If already output, need to remove to add again
        # avoid issues in case you run this cell multiple times
        del test_df[output_tag]

test_df = pd.concat((test_df,pd.DataFrame(
    output,columns=output_tags,index=test_df.index)),axis=1)
# We add the output as a column, a bit messy, different ways, 
# here use a concatenation

# Make the discriminator #
if 'd_HH' in test_df.columns:
    del test_df['d_HH']
    
signal_idx = [i for i,tag in enumerate(tags) if 'HH' in tag]

# d_HH = ln (P(HH) / (P(single H) + P(background)))

#test_df['d_HH'] = pd.Series(np.ones(test_df.shape[0]))

# Numerator #
num = pd.DataFrame((test_df[[output_tags[i]\
     for i in range(len(tags)) if i in signal_idx]]).sum(axis=1))
# Denominator #
den = pd.DataFrame(test_df[[output_tags[i]\
     for i in range(len(tags)) if i not in signal_idx]].sum(axis=1))
# Ln #
d_HH = np.log(num / den)
test_df['d_HH'] = d_HH

if df[df.isin([np.nan, np.inf, -np.inf]).any(1)].shape[0] > 0:
    raise RuntimeError('Some nan of inf values in d_HH')

import roc
importlib.reload(roc) # Reload in case file has changed
for tag in tags:
    print (f'ROC curve of binary classification of {tag}\
         node versus all the others')
    roc.rocAndSig(y_true                 = test_df[tag],
                  y_pred                 = test_df[f'output {tag}'],
                  w_roc                  = test_df['training_weight'],
                  w_sig                  = test_df['event_weight'],
                  show_significance      = 'HH' in tag,
                  outputName             = f'roc_{suffix}_{split}_{tag}.pdf')

# Multiclassification ROC curves are a bit harder to interpret than binary classification
# Here I do one versus the rest, so each ROC curves shows how the DNN is able to classify
# one class (HH, single H or background) versus all the others, which is one projection on
# how to see the performances
# For HH I show the significance but more as an information, because using only the HH node 
# means we do not use all the power of the multiclass (-> d_HH is for that)
print (f'ROC curve of binary classification of d_HH')
roc.rocAndSig(y_true                 = test_df['HH'],
              y_pred                 = test_df['d_HH'],
              w_roc                  = test_df['training_weight'],
              w_sig                  = test_df['event_weight'],
              show_significance      = True,
              outputName             = f'roc_{suffix}_{split}_d_HH.pdf')

# Tryign a new things, seeing the discrimination power of each node, class wise
for tag in tags:
    print (f'Multi roc curve for `output {tag}`')
    tags_order = [tag] + [t for t in tags if t != tag]
    roc.multiRoc(
        outputs    = [test_df[test_df['tag']==tag][f'output {tag}'] \
            for tag in tags_order],
        tags       = tags_order,
        weights    = [test_df[test_df['tag']==tag]['training_weight'] \
            for tag in tags_order],
        title      = f'Using node {tag}',
        outputName = f'multi_roc_{suffix}_{split}_output_{tag}.pdf')

fig,axs = plt.subplots(figsize=(12,25),nrows=len(tags)+1,ncols=2)
fig.subplots_adjust(left=0.1, right=0.9, top=0.98,
                    bottom=0.1, wspace=0.3,hspace=0.5)

tag_df = {tag:test_df[test_df['tag']==tag] for tag in tags}
colors = ['g','r','b']

# Manual binning so we can compute significance #
n_bins = 50

def get_bin_content(bins,y,w):
    digitized = np.digitize(y,bins)
    return np.array([w[digitized==i].sum() for i in range(1, len(bins))])

for irow,output_tag in enumerate(output_tags+['d_HH']):
    for icol,weight in enumerate(['event_weight','training_weight']):
        # Fill the bins myself #
        bins = np.linspace(test_df[output_tag].min(),
                            test_df[output_tag].max(),n_bins+1)
        centers = (bins[1:]+bins[:-1])/2
        widths = np.diff(bins)
        
        tag_content = {tag:get_bin_content(bins,tag_df[tag][output_tag],
                        tag_df[tag][weight])for tag in tags}
        tag_cumsum_left = {tag:np.cumsum(tag_content[tag])/\
            tag_content[tag].sum() for tag in tags}
        tag_cumsum_right = {tag:np.cumsum(tag_content[tag]\
            [::-1])[::-1]/tag_content[tag].sum() for tag in tags}
        # Need to integrate all the bins right of the DNN 
        # cut to get significance
        #z_left = np.nan_to_num(np.sqrt(2*((cumsum_s_left+
        # cumsum_b_left)*np.log(1+cumsum_s_left/cumsum_b_left)
        # -cumsum_s_left)))
        #z_right = np.nan_to_num(np.sqrt(2*((cumsum_s_right+
        # cumsum_b_right)*np.log(1+cumsum_s_right/cumsum_b_right)
        # -cumsum_s_right)))
        #z_left /= z_left.max()
        #z_right /= z_right.max()
        for i,(tag,content) in enumerate(tag_content.items()):
            axs[irow,icol].bar(x=centers,height=content,
            width=widths,fill=False,edgecolor=colors[i],label=tag)     
        #ax2=axs[irow,icol].twinx()   
        
        #ax2.plot(centers,z_left,color='r',label='Significance
        # (left of cut) [normed]')
        #ax2.plot(centers,z_right,color='r',linestyle='--',
        # label='Significance (right of cut) [normed]')

        #for i,tag in enumerate(tag_content.keys()):
        #    ax2.plot(centers,content,color=colors[i],
        # linestyle='-',label=f'{tag} content (left of cut)')
        #    ax2.plot(centers,color=colors[i],
        # linestyle='--',label=f'{tag} content (right of cut)')
        
        #ax2.set_yscale("log")
        #ax2.set_ylim([0,1.4])
        #ax2.set_ylabel('Cumulative distribution')
        #ax2.legend(loc='upper right')

        axs[irow,icol].set_title(f"Using {weight}")
        axs[irow,icol].set_xlabel(output_tag)
        axs[irow,icol].set_ylabel('Yield')
        axs[irow,icol].set_ylim(1e-5,max([content.max() for\
             content in tag_content.values()])*100)
        axs[irow,icol].set_yscale('log')
        axs[irow,icol].legend(loc='upper left')
fig.savefig(f"prediction_{suffix}_{split}.pdf", dpi = 300)

# evaluate the model
scores = model.evaluate(test_df[input_vars], 
                        test_df[tags], 
                        sample_weight = test_df['training_weight'], 
                        batch_size = 5000,
                        verbose=2)
print("%s: %.2f%%" % (model.metrics_names[1], scores[1]*100))

# save model and architecture to single file
modelName = f"model_{suffix}_{split}"
model.save(modelName)
print(f"Saved model to disk as {modelName}")
\end{lstlisting}