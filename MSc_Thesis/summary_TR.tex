2012’de, İsviçre’nin  Cenevre  kenti  ile  Fransa  sınırında  konuşlanmış  Avrupa Nükleer Araştırma Merkezi’nin (CERN) Büyük Hadron Çarpıştırıcısı (LHC) etrafındaki CMS  ve  ATLAS  deneyleri, 125 GeV kütlesinde bir parçacığın gözlendiğini açıkladı. İlk defa varlığı deneyler tarafından kanıtlanan bu parçacık, 1964 yılında Peter Higgs ve 5 diğer bilimadamı tarafından ortaya atılan Higgs bozonuydu. Bir sene sonra Peter Higgs ve François Englert Nobel Fizik Ödülü'ne layık görüldüler. 2012'den itibaren, elektro-zayıf simetri kırınımı mekanizmasını açıklayan ve Standart Model ötesi kuramlar için önemli bir parçacık olan Higgs bozonunun tüm özellikleri detaylı bir şekilde çalışılmaya başlandı.

27 km uzunluğunda çevreye sahip çembersel bir yeraltı tünelini kullanan LHC deneyi, 4 adet büyük dedektörü barındırmaktadır. Bu dedektörler; ATLAS, ALICE, CMS ve LHCb dedektörleridir. Bu dedektörlerden iki tanesi, ATLAS ve CMS, 13 TeV kadar yüksek bir kütle merkezi enerjisinde çarpışan protonlardan saçılan parçacıkları gözlemleyerek, Higgs bozonu dahil olmak üzere diğer tüm temel parçacıkları araştırıyor. LHC deneyi, 2027 yılı sonrası için 14 TeV'lik bir çapışma enerjisine çıkmayı ve toplam lüminositeyi 3000 $fb^{-1}$ değerine kadar yükseltmeyi planlarken CMS deneyi de, çarpışma noktasının etrafındaki kapsama alanını artırmak gibi çeşitli güncellemelerle geliştiriliyor.

Higgs bozonu çifti üretimi LHC'deki proton-proton çarpışmalarında görece yüksek bir kesit alanına sahiptir. 13 TeV enerjide gluon-gluon füzyonu üretim modunda yaklaşık 30 $fb$'dir. Yüksek enerjideki proton-proton çarpışmalarında Higgs bozonu çifti üretiminin önemi ise, Higgs potansiyelinin şeklini belirleyen Higgs tri-lineer self-etkileşme terimine doğrudan erişiminin olmasıdır. LHC deneyinde üretilen Higgs bozonu çiftleri ATLAS ve CMS deneyleri tarafından bir çok bozunma kanalında çalışılmıştır. Fakat Yüksek Lüminosite LHC programı (HL-LHC), yani LHC deneyinin 2027 ve sonrası için planlanan güncellenmiş halindeki Higgs bozonu çifti üretimi henüz tüm bozunma kanallarında çalışılmamıştır.

Bu tezde, HL-LHC programı kapsamında yapılması planlanan güncelleme çalışmaları sonrasındaki durumda Higgs bozonu çifti üretimi, gluon-gluon füzyonu ve vektör bozon füzyonu üretim modları ile CMS dedektörünün Faz-2 için güncellenmiş halinde, proton-proton çarpışması simülasyonları ile çalışılmıştır. 1. bölümde, parçacık fiziğinin başarılı teorisi Standart Model'e giriş niteliğinde açıklamalar verilmiştir. Standar Model Lagrange denklemleri incelenmiş, temel kuvvetler, parçacıklar ve özellikle Higgs bozonundan bahsedilmiştir. Standart Model ötesi araştırmalardan ve bu araştırmalarda Higgs bozonunun öneminden bahsedilmiştir.

2. bölümde, Büyük Hadron Çarpıştırıcısı deneyinin yapısı açıklanmış ve simülasyon programları ile davranışı taklit edilen Kompakt Müon Solenoidi (CMS) dedektörünün özellikleri ve alt bileşenleri incelenmiştir. CMS dedektörünün Faz-2 Güncelleme Çalışmaları ve HL-LHC programı ile ilişkisi ile HL-LHC seviyesindeki Higgs bozonu üretim modları açıklanmıştır.

Higgs bozonu çiftinin \ttgg bozunma kanalındaki analizi  3. bölümde detaylıca incelenmiştir. Öncelikle analiz stratejisinden bahsedilmiş ve simülasyon ile üretilen proton-proton çarpışma verileri; bu verilerin üretiminde kullanılan programlar ve bu programların arka planında çalışan metotlar, sinyal ve ardalan çarpışma proseslerinin detayları ile CMS dedektörünün simülasyonu için kullanılan bilgisayar programının özellikleri açıklanmıştır. Analizde kullanılan fizik objeleri; fotonlar, leptonlar, jetler ve kayıp enerji, CMS dedektörünün kapsama alanı içerisinde ve yeterli enerjiye sahip olacak şekilde tanımlanmıştır. Higgs bozonu çifti üretiminin istenilen kanallarda (\ttgg, \wwgg ve \zzgg) incelenmesini sağlamak amacıyla çarpışma olaylarına belirli kriterler uygulanmıştır. Olay seçimi adı verilen bu işlem ve fizik objesi tanımlamaları, Bamboo adı verilen Python temelli bir analiz kütüphanesi yardımıyla yapılmıştır. Ayrıca fizik objelerini tanımlama performansları hesaplanmıştır.

Tensorflow makine öğrenmesi kütüphanesinin Keras metodu kullanılarak, analiz için bir yapay sinir ağı geliştirilmiştir. Bu çalışmadaki modelin, analizi öğrenip tahminlerde bulunması için fizik objelerinin; olay başına miktarları, dik-eksen momentumları (transverse momentum), dik eksendeki ($\eta$) ve azimut eksenindeki ($\phi$) açı değerleri, kimliklendirme (ID) değerleri, jetler için; alt (\emph{bottom}) kuark kaynaklı olup olmadığını anlamak üzere eklenmiş btag değeri, fotonlar ve tau leptonları için; sabit kütleleri (\emph{invariant mass}), açısal ayrımları (\emph{angular seperation}) gibi değerler girdi olarak verilmiştir. Bu girdileri kullanarak 0 ile 1 arasında yapay zeka ağı puanı dağılımı çıkarılmıştır. Bu dağılımda sinyal ile aradalan proseslerinin birbirinden ayrılması sağlanmış ve yapay zeka ağı puanının farklı aralıklardaki değerleri için kategoriler oluşturulmuştur. Bu kategoriler arasındaki farklar gözlemlenmiş ve en iyi kategori belirlenmiştir. Belirlenen bu kategori analiz kodlamasında baştan bir seçim olarak uygulanmış ve bu seçim sonrasında elde edilen foton çifti sabit kütlesi dağılımı çıkarılmıştır.

Sistematik belirsizlikler 3. bölümün son konusu olarak açıklanmıştır. Bu kısımda CMS ve ATLAS deneyleri tarafından önerilen deneysel ve kesit alanı belirsizlikleri uygulanmıştır. Deneysel belirsizlikler; lüminosite, foton çifti tetiklemesi, foton çifti sabit kütlesi, foton ID, elektron ID, muon ID, tau ID ve jet enerji ölçeği belirsizlikleridir. Kesit alanı belirsizlikleri ise QCD ve parton dağılım fonksiyonları (PDF) olmak üzere iki kısımda her bir proses verisine uygulanmıştır. Farklı olarak sinyal proseslerine top kuark kütlesinin belirsizliği de eklenmiştir. Daha sonra Higgs Combine Tool adı verilen, Higgs analizleri için özelleştirilmiş bir analiz kütüphanesi kullanılarak anlamlılık düzeyleri (significance level) çıkarılmıştır. Higgs Combine Tool kütüphanesine, foton çifti sabit kütlesi dağılımı girdi olarak verilmiş ve sistematik belirsizlikler bu aşamada uygulanmıştır. Çıkan sonuçları teyit etmek amacıyla yakınlık taramaları (\emph{likelihood scan}) yapılmıştır. Higgs bozonu tri-lineer self-etkileşme teriminin, Standard Model'in \kl efektif alan teorisi çerçevesinde taraması yapılmıştır.

Sonuçların sunulduğu 4. bölümde ise, anlamlılık düzeyleri ve yakınlık taramaları verilmiş ve analiz açısından önemleri tartışılmıştır. Tüm son bozunma durumlarındaki ve seçilim yapılmadan önceki olay sayısı tablo halinde verilmiş ve sonuçlar irdelenmiştir. \kl taraması sonucu elde edilen verinin incelemesi yapılmış ve bulunan tüm sonuçların Standart Model beklenen sonuçları ile uyumlu olduğu görülmüştür.